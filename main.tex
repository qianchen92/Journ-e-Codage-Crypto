\documentclass[10pt]{llncs}
\input{preamble.tex}

%%%%%%%%%%%%%%%%%% macros %%%%%%%%%%%%%%%%%%




\begin{document}


\title{Shorter Publicly Verifiable Ciphertexts in Structure-Preserving Chosen-Ciphertext-Secure Public-Key Encryption}

 \author{Beno\^{\i}t Libert\inst{1} \and Thomas Peters\inst{2} \and Chen Qian\inst{3}}
\institute{ CNRS, Laboratoire LIP
  (CNRS, ENSL, U\@. Lyon, Inria, UCBL),\\ ENS de Lyon~(France) \and 
  FNRS \& UCLouvain, ICTEAM~(Belgium) \and Université de Rennes 1 \& IRISA, Rennes (France) }
\maketitle


Structure-preserving cryptography is a world where messages, signatures, ciphertexts and public keys are entirely made of elements of a group over which q bilinear map is efficiently computable.
Especially, this property allows for a smooth interaction of the considered primitives with Groth-Sahai (GS) proof system~\cite{DBLP:journals/eccc/GrothS07}, making them very powerful tools for the modular design of privacy-preserving cryptographic protocols.
While structure-preserving signatures have received much attention the last $6$ years, structure-preserving encryption schemes have undergone slower development.
In particular, the best known  structure-preserving cryptosystems with chosen-ciphertext (IND-CCA2)~\cite{DBLP:conf/pkc/AbeDKNO13} security either rely on 
symmetric pairings or require 
long ciphertexts comprised of hundred of group elements or do not provide publicly verifiable ciphertexts.

We provide a publicly verifiable construction based on the symmetric eXternal Diffie-Hellman (SXDH) assumption in asymmetric bilinear groups
 $e : \G \times \hat{\G} \rightarrow \G_T$, 
which features relatively short ciphertexts.   For typical parameters, our ciphertext size amounts to less than $40$ elements of $\G$. 
As a second contribution, we provide a structure-preserving encryption scheme with perfectly randomizable ciphertexts and replayable 
chosen-ciphertext (RCCA) security which is a meaningful relaxation of CCA2 security for the rerandomizable encryption scheme. 
Our new RCCA-secure system significantly improves upon the best known system featuring similar properties in terms 
of ciphertext size.



\bibliographystyle{alpha}
\bibliography{main} 
\end{document}




