\documentclass[10pt]{llncs}
\usepackage{amssymb}

\usepackage{fullpage}


%\usepackage[top=1.5cm, bottom=1.5cm, left=1.5cm, right=1.5cm]{geometry}
\usepackage[english]{babel}
\usepackage[utf8]{inputenc}
  
%\usepackage[T1]{fontenc}
\usepackage{lmodern}
\usepackage[onelanguage,boxed]{algorithm2e}
\usepackage{graphicx}
\usepackage{float}
\usepackage{amsmath}
\usepackage{amsfonts}
\usepackage{amsthm}
\usepackage{color}
\usepackage[usenames,dvipsnames]{xcolor}
\usepackage[toc,page]{appendix}
\usepackage{listings}
\usepackage{tikz}
\usepackage{subfigure}
\usepackage{wrapfig}
\usepackage{hyperref}
\hypersetup{colorlinks,linkcolor=black,urlcolor=blue}
\usepackage[font=small,labelfont=bf]{caption}
\newtheorem {definition} {Definition}
\newtheorem {construction} {Construction}
\newtheorem {lemma}{Lemma}

\lstset
{
  emph={Stack},
  emphstyle={\color{DarkOrchid}}
}


\newcommand{\OTS}{\mathcal{OTS}}
\newcommand{\TC}{\mathcal{TC}}
\newcommand{\Adv}{\mathbf{Adv}}

\newcommand{\A}{\mathcal{A}}
\newcommand{\B}{\mathcal{B}}
\newcommand{\D}{\mathcal{D}}
\newcommand{\G}{\mathbb{G}}
\newcommand{\Z}{\mathbb{Z}}
\newcommand{\PR}{\operatorname{Pr}}
\newcommand{\PP}{\mathsf{P}}  
\newcommand{\VV}{\mathsf{V}}  
\newcommand{\K}{\mathsf{K}}  
\newcommand{\SIM}{\mathsf{S}}  
\newcommand{\lbl}{\mathsf{lbl}} 
\newcommand{\PPE}{\mathrm{PPE}} 
\newcommand{\SK}{\mathsf{SK}}
\newcommand{\PK}{\mathsf{PK}}
\newcommand{\VK}{\mathsf{VK}}
\newcommand{\SSK}{\mathsf{SSK}}
\newcommand{\SVK}{\mathsf{SVK}}
\newcommand{\sk}{\mathsf{sk}}
\newcommand{\ck}{\mathsf{ck}}
\newcommand{\tk}{\mathsf{tk}}
\newcommand{\msk}{\mathsf{msk}}
\newcommand{\vk}{\mathsf{vk}}
\newcommand{\ovk}{\mathsf{ovk}}
\newcommand{\pk}{\mathsf{pk}}
\newcommand{\opk}{\mathsf{opk}}
\newcommand{\osk}{\mathsf{osk}}
\newcommand{\com}{\mathsf{com}}
\newcommand{\open}{\mathsf{open}}
\newcommand{\True}{\mathsf{True}}
\newcommand{\False}{\mathsf{False}}
\newcommand{\BF}{\mathbf}
\newcommand{\sample}{\stackrel{{\scriptscriptstyle \mkern4mu R}}{\gets}}
\newcommand{\etal}{\textrm{el. al.}}
\newcommand{\etc}{\textrm{etc.}}
\newcommand{\eg}{\textrm{e.g.} }
\newcommand{\ie}{\textrm{i.e.} }
\newcommand{\wrt}{\textrm{w.r.t.} }
\newcommand{\st}{\textrm{s.t.} }
\newcommand{\resp}{\textrm{resp.} }
\providecommand{\tprod}{{\textstyle\prod}}
\newcommand{\Setup}{{\mathsf{Setup}}}
\newcommand{\SKGen}{{\mathsf{SecretKeyGen}}}
\newcommand{\PKGen}{{\mathsf{PublicKeyGen}}}
\newcommand{\KeyGen}{{\mathsf{KeyGen}}}
\newcommand{\OKeyGen}{{\mathsf{OKeyGen}}}
\newcommand{\Enc}{{\mathsf{Encrypt}}}
\newcommand{\Dec}{{\mathsf{Decrypt}}}
\newcommand{\MPDec}{{\mathsf{MPDecrypt}}}
\newcommand{\Rerand}{{\mathsf{ReRand}}}
\newcommand{\Sig}{{\mathsf{Sign}}}
\newcommand{\Verif}{{\mathsf{Verify}}}
\newcommand{\Prove}{{\mathsf{Prove}}}
\newcommand{\Com}{{\mathsf{Commit}}}
\newcommand{\PPP}{{\mathsf{PP}}}
\newcommand{\Forge}{{\mathsf{Forge}}}

%GSW encryption components
\newcommand{\BitDecomp}{{\mathsf{BitDecomp}}}
\newcommand{\Flatten}{{\mathsf{Flatten}}}
\newcommand{\Powers}{{\mathsf{PowersOf2}}}



\usepackage{mathtools}
\DeclarePairedDelimiter{\ceil}{\lceil}{\rceil}
\DeclarePairedDelimiter{\round}{\lfloor}{\rceil}
\DeclarePairedDelimiter{\floor}{\lfloor}{\rfloor}
%\newcommand{\Adv}{\mathcal{A}}


%%--- Page settings
\makeatletter
\renewcommand{\paragraph}{%
  \@startsection {paragraph}{4}{0pt}{-12pt \@plus -4pt \@minus -4pt}%
  {-0.5em \@plus -0.22em \@minus -0.1em}{\normalfont\normalsize\scshape }}
\makeatother
% \usepackage{enumitem} 
% \setlist[description]{itemsep=2pt}
 




%%%%%%%%%%%%%%%%%%%%%%%%%%%%%%%%%%%%%%%%%%%%%



\usepackage{comment}



%%%% TODO package
\usepackage{lipsum}                     % Dummytext
\usepackage{xargs}                      % Use more than one optional parameter in a new commands
\usepackage{xcolor}  % Coloured text etc.
\usepackage[colorinlistoftodos,prependcaption,textsize=tiny]{todonotes}

\newcommandx{\unsure}[2][1=]{\todo[linecolor=red,backgroundcolor=red!25,bordercolor=red,#1]{#2}}
\newcommandx{\change}[2][1=]{\todo[linecolor=blue,backgroundcolor=blue!25,bordercolor=blue,#1]{#2}}
\newcommandx{\info}[2][1=]{\todo[linecolor=OliveGreen,backgroundcolor=OliveGreen!25,bordercolor=OliveGreen,#1]{#2}}
\newcommandx{\improvement}[2][1=]{\todo[linecolor=Plum,backgroundcolor=Plum!25,bordercolor=Plum,#1]{#2}}
\newcommandx{\thiswillnotshow}[2][1=]{\todo[disable,#1]{#2}}


%%%%%%%%%%%%%%%%%% macros %%%%%%%%%%%%%%%%%%




\begin{document}


\title{Shorter Publicly Verifiable Ciphertexts in Structure-Preserving Chosen-Ciphertext-Secure Public-Key Encryption}

 \author{Beno\^{\i}t Libert\inst{1} \and Thomas Peters\inst{2} \and Chen Qian\inst{3}}
\institute{ CNRS, Laboratoire LIP
  (CNRS, ENSL, U\@. Lyon, Inria, UCBL),\\ ENS de Lyon~(France) \and 
  FNRS \& UCLouvain, ICTEAM~(Belgium) \and Université de Rennes 1 \& IRISA, Rennes (France) }
\maketitle


Structure-preserving cryptography is a world where messages, signatures, ciphertexts and public keys are entirely made of elements of a group over which q bilinear map is efficiently computable.
Especially, this property allows for a smooth interaction of the considered primitives with Groth-Sahai (GS) proof system~\cite{DBLP:journals/eccc/GrothS07}, making them very powerful tools for the modular design of privacy-preserving cryptographic protocols.
While structure-preserving signatures have received much attention the last $6$ years, structure-preserving encryption schemes have undergone slower development.
In particular, the best known  structure-preserving cryptosystems with chosen-ciphertext (IND-CCA2)~\cite{DBLP:conf/pkc/AbeDKNO13} security either rely on 
symmetric pairings or require 
long ciphertexts comprised of hundred of group elements or do not provide publicly verifiable ciphertexts.

We provide a publicly verifiable construction based on the symmetric eXternal Diffie-Hellman (SXDH) assumption in asymmetric bilinear groups
 $e : \G \times \hat{\G} \rightarrow \G_T$, 
which features relatively short ciphertexts.   For typical parameters, our ciphertext size amounts to less than $40$ elements of $\G$. 
As a second contribution, we provide a structure-preserving encryption scheme with perfectly randomizable ciphertexts and replayable 
chosen-ciphertext (RCCA) security which is a meaningful relaxation of CCA2 security for the rerandomizable encryption scheme. 
Our new RCCA-secure system significantly improves upon the best known system featuring similar properties in terms 
of ciphertext size.



\bibliographystyle{alpha}
\bibliography{main} 
\end{document}




